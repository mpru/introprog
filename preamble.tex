% \usepackage{booktabs}
% \usepackage{amsthm}
% \usepackage{amssymb}
\usepackage{booktabs}
\usepackage{multirow}

\usepackage[spanish]{babel}
\usepackage[utf8]{inputenc}
\spanishdecimal{.}
%\renewcommand{\contentsname}{Contenido}


% Encabezado
\usepackage[margin=2cm]{geometry}
\usepackage{fancyhdr}
\pagestyle{fancy}
\renewcommand{\footrulewidth}{0.4pt}
\renewcommand{\sectionmark}[1]{\markright{#1}}
\fancyhead[L]{IALP - Guía de Estudio}
\fancyhead[R] {\leftmark}

% Formato para el capítulo
% Para que diga Unidad en lugar de Capítulo
\makeatletter
\renewcommand{\@chapapp}{Unidad}
\makeatother
% Para que diga "Unidad" antes del nro y nombre de la unidad
%\usepackage{titlesec}
%\titleformat{\chapter}{
%	\normalfont\LARGE\bfseries
%}{Unidad \thechapter.  }{0em}{}

% Para que haya encabezado en la primera hoja de cada cap tmb
\usepackage{etoolbox}
\patchcmd{\chapter}{\thispagestyle{plain}}{\thispagestyle{fancy}}{}{}

% Pero lo anterior agrega encabezad en la tabla de contenidos, esto lo evita
\AtBeginDocument{%
	\addtocontents{toc}{\protect\thispagestyle{empty}}
}

% Agregar puntos en el TOC en el nivel de capitulo
\usepackage{tocloft}
\renewcommand{\cftchapleader}{\cftdotfill{\cftdotsep}} % for chapters

% Info para la portada
\newcommand{\horrule}[1]{\rule{\linewidth}{#1}}
\title{
	%\vspace{-1in} 	
	\usefont{OT1}{bch}{b}{n}
	\normalfont \normalsize
	\textsc{
		Universidad Nacional de Rosario \\
		Facultad de Ciencias Económicas y Estadística \\
		Licenciatura en Estadística
	} \\ [25pt]
	\horrule{2pt} \\[0.4cm]
	\huge \textbf{Introducción a la Programación} \\
	\bigbreak
	Guía de Estudio - Año 2021\\
	\horrule{2pt} \\[0.5cm]}

\author{
	\normalfont Mgs. Lic. Marcos Prunello (Prof. Tit. Ordenado)
}

% Esto es para que el cuadro sombreado donde sale el codigo sea mas chico y para que el espacio entre el codigo y el output sea menor
\usepackage{etoolbox,framed} 
\setlength{\parskip}{2pt}
\setlength{\OuterFrameSep}{2pt}
\makeatletter
\preto{\@verbatim}{\topsep=2pt \partopsep=2pt }
\makeatother

% Otros paquetes
\usepackage{xcolor}

% cosas para modificar espacios entre ecuaciones y otros
% Que haya menos espacio antes y dps de ecuaciones
% \usepackage{setspace}\onehalfspacing
% \AtBeginDocument{%
%   % estas dos 
%   %\addtolength\abovedisplayskip{-0.5\baselineskip}%
%   %\addtolength\belowdisplayskip{-0.5\baselineskip}
%   % o estas dos, probar cual me gusta mas
%   \setlength{\belowdisplayskip}{0pt} \setlength{\belowdisplayshortskip}{0pt}
%   \setlength{\abovedisplayskip}{0pt} \setlength{\abovedisplayshortskip}{0pt}
% }
% 
% \setlength{\intextsep}{0pt}
% 
% \makeatletter
% \def\thm@space@setup{%
%   \thm@preskip=8pt plus 2pt minus 4pt
%   \thm@postskip=\thm@preskip
% }
% \makeatother

% Columnas
\newenvironment{columns}[1][]{}{}

\newenvironment{column}[1]{\begin{minipage}{#1}\ignorespaces}{%
\end{minipage}
\ifhmode\unskip\fi
\aftergroup\useignorespacesandallpars}

\def\useignorespacesandallpars#1\ignorespaces\fi{%
#1\fi\ignorespacesandallpars}

\makeatletter
\def\ignorespacesandallpars{%
  \@ifnextchar\par
    {\expandafter\ignorespacesandallpars\@gobble}%
    {}%
}
\makeatother

% Para que en Latex las tables queden donde quiero
\usepackage{float}
% \floatplacement{table}{H}